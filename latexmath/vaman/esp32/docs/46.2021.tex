\def\mytitle{IMPLEMENTATION OF RANDOM NUMBER GENERATOR IN VAMAN ESP}
\def\mykeywords{}
\documentclass[10pt,a4paper]{article}
\usepackage[a4paper,outer=1.5cm,inner=1.5cm,top=1.75    cm,bottom=1.5cm]{geometry}
%  \twocolumn
\usepackage{graphicx}
\usepackage{amsfonts}
\usepackage{circuitikz}
\usepackage[colorlinks,linkcolor={black},citecolor={blue!80!black},urlcolor={blue!80!black}]{hyperref}
\usepackage[parfill]{parskip}
\usepackage{lmodern}
\usepackage{tabularx}
\usepackage{tikz}
%\usepackage{geometry}
%\usetikzlibrary{shapes,arrows,chains,decorations.markings,intersections,calc}
\usetikzlibrary{positioning}
\usepackage{xcolor}
\usepackage{multirow}
\usepackage{listings}
\usepackage{float}
\usepackage{titlesec}
\usepackage{amsmath}
\usepackage[utf8]{inputenc}
\usepackage{algorithm2e}
\usepackage{karnaugh-map}
\usepackage{enumitem}
\usepackage{lipsum}
\usepackage{array}
\usepackage{longtable}
\usepackage{hhline}
\usepackage{ifthen}
\usepackage{datetime}
\usepackage{amsmath}
\usepackage{textgreek}
\usepackage{tikz}
\usetikzlibrary{calc}                         
\usetikzlibrary{circuits.logic.US}
\title{\mytitle}
 \author{MARIKUNDAM HARSHITHA\\marikundamdec@gmail.com\\FWC22120 IITH-Future Wireless Communications     Assignment-6}
\date{}
\sloppy
\lstset{                                          
language=C++,                           
basicstyle=\ttfamily\footnotesize,   
breaklines=true,                       
frame=lines
}

\begin{document}
\maketitle
\tableofcontents

\section{Problem}
(GATE2021-QP-EC)\\
Q.46 The propogation delay of the exclusive-OR(XOR) gate in the circuit in the figure is 3ns.The propogation delay of all the flip-flops is assumed to be zero.The clock(Clk) frequency provided to the circuit is 500MHz.\\
\begin{figure}[!h]
\begin{center}
\resizebox{0.5\columnwidth}{!}{
\input{figs/fig1.tex}
}
\end{center}
	\caption{Circuit}
\label{fig:circuit}
\end{figure}

Starting from the initial value of the flip-flop outputs $Q2Q1Q0 =111$ with $D2=1$,the minimum number of triggering clock edges after which the flip-flop outputs $Q2Q1Q0$ becomes 1 0 0\emph{(in integer)} is \line(1,0){12.5}

\section{Introduction}
A random number generator using D flip-flops is a simple digital circuit that generates a sequence of random binary numbers.To implement this type of random number generator, we use a series of D flip-flops connected in a feedback loop. The output of each flip-flop is fed back into the input of the next flip-flop,creating a circuit that generated a sequence of random binary values.\\ \\
The feedback loop creates a delay in the circuit,which causes the circuit to exhibit unpredictable behavior.This unpredictable behavior results in a sequence of random binary values. The length of the delay can be adjusted to control the randomness of the output.

\section{Components}
\begin{table}[!h]
\centering
\begin{tabular}{|p{4cm}|p{3cm}|p{3cm}|}
\hline                                        
\textbf{Components} & \textbf{Value} & \textbf{Quantity}\\                                          
\hline                                 
Breadboard &      & 1 \\           
\hline                                    
USB-C Cable &  & 1 \\     
\hline                      
Vaman &  & 1 \\       
\hline                                     
Seven Segment Display & Common Anode & 1\\
\hline                      
Decoder & 7447 & 1 \\        
\hline                     
Flip Flop & 7474 & 2 \\
\hline                        
Jumper Wires &    & 39\\   
	\hline
\end{tabular}

\caption{Components}
\label{table:components}
\end{table}
\pagebreak
\subsection{Seven Segment Display}
The seven segment display has eight pins, \emph{a,b,c,d,e,f,g} and \emph{dot} that take an active LOW input,i.e. the LED will glow only if the input is connected to ground.Each of these pins is connected to an LED segment.The \emph{dot} pin is reserved for the LED.
\section{Setup}
\begin{enumerate}
\item Connect the Vaman to the Laptop through USB.
\item There is a button and an LED to the left of the USB port on the Vaman.There is another button to the right of the LED.
\item Press the right button first and immediately press the left button.The LED will be blinking green.The Vaman is now in bootloader mode.
\end{enumerate}
\subsection{Steps for Implementation}
\begin{enumerate}
\item Connect the USB-UART pins to the Vaman ESP32 pins according to Table
\begin{table}[!h]                             
\centering                       
\begin{tabular}{|p{4cm}|p{4cm}|}    
\hline              
\textbf{VAMAN LC PINS} & \textbf{UART PINS}\\                                  
\hline                     
GND & GND \\                 
\hline                        
ENBe & ENB \\         
\hline                
TXD0 & RXD \\              
\hline            
RXD0 & TXD\\   
\hline                                  
0 & IO0 \\                  
\hline                               
5V & 5V\\         
\hline                    
\end{tabular}
               
\caption{Set Up}              
\label{table:setup}      
\end{table}
 \item Flash the following setup code through USB-UART using laptop
\begin{center}
\fbox{\parbox{8cm}{\url{https://github.com/Marikundam/fwc/tree/main/latexmath/vaman/esp32/codes/src/code6.cpp}}}
\end{center}
\begin{center}
\end{center}
\begin{lstlisting}
svn co https://github.com/Mariku    ndam/fwc/tree/main/latexmath/vaman/esp32/codes/setup
cd  setup
pio run
pio run -t upload
\end{lstlisting}
after entering your wifi username and password (in quotes below)
\begin{lstlisting}
#define STASSID "..." // Add your network credentials
#define STAPSK  "..."
\end{lstlisting}
in src/code6.cpp file
\item You can notice that vaman will be connnected to the network credentials provided above.Connect your laptop to the same network ,You should be able to find the ip address of your vaman-esp on laptop using 
\begin{lstlisting}
ifconfig
nmap -sn 192.168.85.209
\end{lstlisting}
where your computer's ip address is the output of ifconfig and given by 192.168.85.x
\item Login to termux-ubuntu on the android device and execute the following commands:
\begin{lstlisting}
proot-distro login debian
cd  /data/data/com.termux/files/home/
mkdir iot
svn co https://github.com/Marikundam/fwc/tree/main/latexmath/vaman/esp32/codes
cd codes
\end{lstlisting}
\item Assuming that the username is harshitha and password is marikundam, flash the following code wirelessly
\begin{center}
\fbox{\parbox{8cm}{\url{https://github.com/Marikundam/fwc/tree/main/latexmath/vaman/esp32/codes/src/code6.cpp}}}
\end{center}
through 
\begin{lstlisting}
pio run 
pio run -t nobuild -t upload --upload-port ip$_$addres$_$of$_$esp
\end{lstlisting}
where you may replace the above ip address with the ip address of your vaman-esp.
\end{enumerate}
\section{Implementation}
A 7474 IC which  has 14 pins and can store two seperate binary values.So we consider two IC's since we have three values  and connect the  D inputs of each flip-flop to the input signals of 7447 IC . Later interface 7447 IC to seven segment display for the output. The CLK input is used to trigger the flip-flop,and the Q output is used to read the stored value.When a positive edge is detected on the CLK input,the current value on the D input is stored in the flip-flop. The boolean expression of the D flip-flop is $Q(t+1) = D$
\subsection{Truth table}
\begin{table}[!h]
\centering
\begin{tabular}{|p{3cm}|p{1cm}|p{1cm}|p{1cm}|p{1cm}|p{1cm}|p{1cm}|p{1cm}|}                            
\hline                                         
7447& $\overline{a}$ & $\overline{b}$ & $\overline{c}$ & $\overline{d}$ & $\overline{e}$ & $\overline{f}$ & $\overline{g}$\\          
\hline                                          
Display& a& b& c& d& e& f& g\\         
\hline                                        
\end{tabular}
   
\caption{Truth Table}
\label{table:truth_table}
\end{table}
\pagebreak
\subsection{K-map}
Since $Q'= D$,we find the k-maps for D as outputs\\
\begin{figure}[!h]                                
\begin{center}                                 
\resizebox{0.5\columnwidth}{!}{
\input{figs/kmap1.tex}
}                                                
\end{center}                                     
\caption{For D2}                                       
\label{fig:for_D2}                             
\end{figure}
%
\begin{figure}[!h]                              
\begin{center}                                 
\resizebox{0.5\columnwidth}{!}{
\input{figs/kmap2.tex}
}                                               
\end{center}                                   
\caption{For D1}                                     
\label{fig:for_D1}                             
\end{figure}
%
\begin{figure}[!h]                             
\begin{center}                                
\resizebox{0.5\columnwidth}{!}{
\input{figs/kmap3.tex}
}                                              
\end{center}                              
\caption{For D0}                                
\label{fig:for_D1}                            
\end{figure}    
\pagebreak
%
\subsection{Boolean Equation}
By solving the K-maps above we obtain as follows :
\begin{align}
	D2 &= \overline{Q2}Q0 + \overline{Q0}Q2 \\
	D1 &= Q2 \\
	D0 &= Q1 
\end{align}
\section{Hardware}
\begin{enumerate}
\item Make the connections between the seven segment display and the 7447 IC as shown in Table3
\begin{table}[!h]                                
\centering
\begin{tabular}{|p{3cm}|p{1cm}|p{1cm}|p{1cm}|p{1cm}|}                                                  
\hline                                     
7447& D& C& B& A\\               
\hline                      
Arduino& 5& 4& 3& 2\\       
\hline                                       
\end{tabular}
   
\caption{7447}                               
\label{table:7447}                       
\end{table}
\item Connect the Vaman,7447 and the two 7474 ICs according to Table4
\begin{table}[!h]                                 
\centering	
\begin{tabular}{|c|c|c|c|c|c|c|c|c|c|c|c|c|}      
\hline                              
\multirow{2}{*}{} & \multicolumn{3}{|c|}{INPUT} & \multicolumn{3}{|c|}{OUTPUT} & \multicolumn{2}{|c|}{\multirow{2}{*}{CLOCK}} & \multicolumn{4}{|c|}{\multirow{3}{*}{5V}} \\      
\cline{2-7}     
& Q0 & Q1 & Q2 & Q0' & Q1' & Q2' & \multicolumn{2}{|c|}{\multirow{2}{*}{}} & \multicolumn{4}{|c|}{} \\        
\hline          
Vaman & IO4 & IO5 & IO6 & IO1 & IO2 & IO3 & \multicolumn{2}{|c|}{IO7} & \multicolumn{4}{|c|}{\multirow{3}{*}{}}\\                                   
\hline                             
7474 & 5 & 9 &  & 2 & 12 &  & CLK1 & CLK2 & 1 & 4 & 10 & 13 \\                   
\hline                     
7474 & & & 5 & & & 2 & CLK1 & CLK2 & 1 & 4  & 10 & 13 \\                       
\hline                         
7447 & \multicolumn{3}{|c|}{} & 7 & 1 & 2 & & & \multicolumn{4}{|c|}{16} \\              
\hline
\end{tabular}
 
\caption{Connections}                                   
\label{table:connections}                       
\end{table}
\item Make the other D input pins of 7474 grounded and supply  5V and GND from the Vaman as well.
\item When the clock edge is trigerred we observe display of random numbers.
\end{enumerate}
\section{Software}
Now write the following code and upload in vaman to see the results.
\lstinputlisting{code6.cpp}
\end{document}
